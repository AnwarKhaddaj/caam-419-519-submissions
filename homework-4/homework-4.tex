\documentclass{article}

\usepackage{graphicx}
\usepackage{layout}
\usepackage{float}
\usepackage{listings}

\title{CAAM 519, Homework $\#4$}
\author{\texttt{ask15}}
\date{November 30, 2022}
\begin{document}
\maketitle

\section{main.cpp file and its output}
\subsection{Main.cpp file}
\begin{lstlisting}
  #include <iostream>
  #include ``vector.h''
  #include ``matrix.h''
  using namespace std;

  int main(void){
    Matrix<double> A(6,4);
    Matrix<double> B(4,5);
    Matrix<double> C(6,5);
    Vector<double> z(5);

    for (int i = 0; i < A.num_rows(); ++i){
      for (int j = 0; j < A.num_columns(); ++j){
        A[i][j] = (double) (i+j);
      }
    }
    for (int i = 0; i < B.num_rows(); ++i){
      for (int j = 0; j < B.num_columns(); ++j){
        B[i][j] = (double) 1 / (i + j + 1);
      }
    }
    for (int i = 0; i < C.num_rows(); ++i){
      for (int j = 0; j < C.num_columns(); ++j){
        C[i][j] = (double) i * j;
      }
    }
    Vector<double> x(5);
    for (int i = 0; i < x.length(); ++i){
      x[i] = i;
    }
    Vector<double> y(6);
    for (int i = 0; i < y.length(); ++i){
      y[i] = 1 - i;
    }

    double a = 1.5;
    std::cout << ``Matrix A is:'' << std::endl;
    A.print();
    std::cout << ``Matrix B is:'' << std::endl;
    B.print();
    std::cout << ``Matrix C is:'' << std::endl;
    C.print();
    std::cout << ``Vector x is:'' << std::endl;
    x.print();
    std::cout << ``Vector y is:'' << std::endl;
    y.print();
    std::cout << ``Scalar a is: `` << a << std::endl;
    z=((A*B + C) * x + a * y);
    std::cout << ``First computation of vector z is:'' << std::endl;
    z.print();
    z = 3.0*z - (y - 1.0) / 2.0 + 0.5;
    std::cout << ``Second computation of vector z is:'' << std::endl;
    z.print();
  }
\end{lstlisting}
\subsection{Commands to get the output}
\begin{lstlisting}
  g++ --std=c++11 -o main -I./include main.cpp
  ./main
\end{lstlisting}
\subsection{Output of main.cpp file}
\begin{lstlisting}
  Matrix A is:
  Matrix = [
    0 1 2 3
    1 2 3 4
    2 3 4 5
    3 4 5 6
    4 5 6 7
    5 6 7 8
  ]
  Matrix B is:
  Matrix = [
    1 0.5 0.333333 0.25 0.2
    0.5 0.333333 0.25 0.2 0.166667
    0.333333 0.25 0.2 0.166667 0.142857
    0.25 0.2 0.166667 0.142857 0.125
  ]
  Matrix C is:
  Matrix = [
    0 0 0 0 0
    0 1 2 3 4
    0 2 4 6 8
    0 3 6 9 12
    0 4 8 12 16
    0 5 10 15 20
  ]
  Vector x is:
  Vector = [
    0
    1
    2
    3
    4
  ]
  Vector y is:
  Vector = [
    1
    0
    -1
    -2
    -3
    -4
  ]
  Scalar a is: 1.5
  First computation of vector z is:
  Vector = [
    11.4286
    47.9286
    84.4286
    120.929
    157.429
    193.929
  ]
  Second computation of vector z is:
  Vector = [
    34.7857
    144.786
    254.786
    364.786
    474.786
    584.786
  ]
\end{lstlisting}
\section{How to overload += operators}
It is very similar to the combination of overloading operators + and then = all in one method.First, we create a vector of certain matching length and then add the corresponding elements. After that, we do the same as we did for overloading the = operator.\\
This method is more efficient as it would cost less than storing a new vector and then returning this vector before resizing it and setting it equal to another vector.
\end{document}
